%%%%%%%%%%%%%%%%%%%%%%%%%%%%%%%%%%%%%%%%%%%%%%%%%%%%%%%%%%%%%%%%%%%%%%
%                                                                    %
% 土木学会論文集マニュアル                                           %
% jsce.sty                                                           %
%                                                                    %
%%%%%%%%%%%%%%%%%%%%%%%%%%%%%%%%%%%%%%%%%%%%%%%%%%%%%%%%%%%%%%%%%%%%%%
\typeout{^^J****************************************************}
\typeout{** 土木学会論文集スタイルファイル用マニュアル  **}
\typeout{** アスキー pLaTeX で二回コンパイルしてください.**}
\typeout{** マニュアル作成には jsce.sty を使います.   **}
\typeout{****************************************************^^J}
%

\documentclass[onecolumn]{jsce}  % for LaTeX2e users 
%
\volumenumber{207}
\pubyear{2161}
\receivedate{2001年9月30日}
\title{土木学会論文集作成マニュアル}
\endtitle{A Manual of JSCE Style Files for \LaTeX}
\author{土木学会論文集編集委員会\thanks{東京都新宿区四谷一丁目無番地}}
\endauthor{JSCE Committee}
\abstract{土木学会論文集の原稿を作成するためのマニュアル.}
\keywords{style files, \LaTeX, Japan Society of Civil Engineers Journals}
\unitlength=1mm
\def\subpicture#1{\begin{picture}(50,7)(0,0)
\put(0,0){\framebox(50,7){#1}}\end{picture}}
% \pagestyle{plain}
\begin{document}
\maketitle

\tableofcontents

\section{\LaTeX のバージョン}

このスタイルファイル群はすべて\LaTeX 2e用である.\LaTeX 209を
使う場合には,それ専用のスタイルファイルを使用して欲しい.
なお,アスキーのp\LaTeX を前提としてある.
角藤版の配布アーカイブでは \verb+platex+ によるコンパイル用である.
本バージョンからNTT J\TeX 用は配布していないので注意する.

\section{句読点についての注意}
\label{sec:1}

横書きの組版規則に関連して
土木学会規程では,句読点には全角の`,'と`.'を用いる.
ただし,文部科学省は`,'と`。'を用いるように指導しているので,
教科書等ではそうなっているのが多いので注意する.

\section{学会論文集の基本的な書き方}

使うスタイルファイルは{\tt jsce.sty}である.2段組なので
次のように書き出す.

\renewcommand{\baselinestretch}{0.75}\small\normalsize
\begin{verbatim}
     \documentclass{jsce} % for LaTex2e users
     % \inenglish         % 英語で書く場合にはアンコメント
     % \finalversion      % 最後の最後はヘッダが不要らしい
     %
     \volumenumber{9}     % 英語論文の場合の巻番号 Vol. 
     \journalnumber{4}    % 英語論文の場合の論文番号 No.
                          % ここの例なら Vol.9, No.4 となる
     \pubmonth{5}         % 発行される月
     \pubyear{1993}       % 発行年
     \procnumber{400}     % 和文の場合の論文集番号 No.
     \procnumberofthefield{22}  %  和文の場合の部門毎の論文番号
                          % ここの例なら No.400/I-22 となる
     \firstpagenumber{5}  % 最初の頁番号
     \firstenglishpagenumber{11} % 英文論文集の場合の最初の頁番号
     \fieldnumber{1}      % 部門番号 (7部門には未対応)
\end{verbatim}
\renewcommand{\baselinestretch}{1}\small\normalsize
こうした上で表題および最後の部分を設定し,本文につなぐ.

\renewcommand{\baselinestretch}{0.75}\small\normalsize
\begin{verbatim}
     \title{メインのタイトル:和文なら日本語}
     \endtitle{別言語のタイトル:和文なら英語}
     \author{著者1\thanks{正会員 学位 職 所属(住所)}・
             著者2\thanks{....}}
     \endauthor{別言語の著者リスト}
     \abstract{概要....空行可}   % 空行入れれば段落
     \keywords{キーワード}
     \endabstract{別言語の概要...空行可}
     % \titlepagecontrol{1} % メインタイトル部分と本文との間隔調整(通常不要)
     \receivedate{1994. 2. 15} % 受理された年月日.英文なら January 15, 1991
     %
     \begin{document}
     \maketitle
\end{verbatim}
\renewcommand{\baselinestretch}{1}\small\normalsize
これから下はすべて本文.\verb+\section+で始める.
章節建ては \verb+\subsubsection+までの3階層で区別する.
ごく標準の\LaTeX で執筆すること.
特殊なマクロ等は可能な限り使わない.

\begin{figure}[htb]
\begin{center}
\setlength{\unitlength}{1mm}
\begin{picture}(110,65)(0,0)
\put(5,5){\thicklines\framebox(40,55){}}
\put(60,5){\thicklines\framebox(40,55){}}
\put(10,10){\dashbox{0.3}(13,45){}}
\put(27,38){\dashbox{0.3}(13,17){}}
\put(65,24){\dashbox{0.3}(13,31){}}
\put(82,24){\dashbox{0.3}(13,31){}}
\put(65,15){\dashbox{0.5}(30,7){他言語タイトル等}}
\put(5,60){\makebox(40,5){最終頁}}
\put(60,60){\makebox(40,5){最終頁}}
\put(5,0){\makebox(40,5){調整前}}
\put(60,0){\makebox(40,5){調整後}}
\put(48,35){\vector(1,0){9}}
\put(30,20){\vector(0,1){17}}
\put(30,20){\vector(0,-1){10}}
\put(66,15){\vector(0,1){7}}
\put(66,15){\vector(0,-1){5}}
\put(30,20){\makebox(15,5){$x$ cm}}
\put(67,5){\makebox(25,10)[l]{$x/2$ cm}}
\put(82,24){\makebox(13,5)[r]{\footnotesize (受理)}}
\put(27,38){\makebox(13,5)[r]{\footnotesize (受理)}}
\put(66,50){\makebox(13,5)[l]{\footnotesize {\tt \symbol{'134}lastpage-}}}
\put(67,47){\makebox(13,5)[l]{\footnotesize {\tt control\{x/2\}}}}
\put(28,50){\makebox(13,5)[l]{\footnotesize 4. 結論}}
\put(83,36){\makebox(13,5)[l]{\footnotesize 4. 結論}}
\end{picture}
\caption{最終頁レイアウト調整の概略}
\label{fig:layout}
\end{center}
\end{figure}

論文の最後の頁には他言語によるタイトル他が配置される.
そのためにはまず1回コンパイルし,\figno{\ref{fig:layout}}の
左図のように右段の下空白部の高さを測定する.
そうした上で左段の上方にくる文章のどこかに
\begin{verbatim}
     \lastpagecontrol{8cm}
\end{verbatim}
といった調整コマンドを挿入する.
これで,この頁の下に他言語タイトル等が出力される.
この高さをどのくらいにすると美しいかについては,trial and errorで
調整する.
もし本文との間にさらに空白が必要な場合には
\begin{verbatim}
     \lastpagecontrol[5mm]{8cm}
\end{verbatim}
のように追加スペースを指定する.
この追加分は他言語タイトル等の箱と本文との間のスペースである.
この空白部分にタイトル等が入りきらない場合には,
次のコマンドで自動的に次頁の頭に出力される.

最後の最後は受理年月日を出力するために,本文の最後の行に
\begin{verbatim}
     \lastpagesettings
\end{verbatim}
というコマンドを挿入する.
ただし,もし上の \verb+\lastpagecontrol+で他言語タイトル等が
入りきらない場合には,この \verb+\lastpagesettings+で
次の頁の頭に他言語タイトル等が設定される.
このとき,タイトル等よりも上方の空白を
\begin{verbatim}
     \lastpagesettings[3cm]
\end{verbatim}
のようにも指定できる.

\section{標準以外に追加した新しいコマンド・環境等}

\subsection{フロートや表}

\begin{Description}
%
\item[ 写真環境:] \verb+\begin{photo}+で始まる環境で,{\tt figure},
 {\tt table}環境と同じフロート
%
\item[ 図表番号の引用:] 本文中で図表番号を引用する場合には,
例えば図なら \verb+\figno{\ref{fig:setting}}+等が使える.
ただし漢字とのスペースの関係で,次のように使い分ける.
\begin{Description}
\item[ a) すぐうしろが漢字の場合:] \verb+\figno{\ref{fig:layout}}+を
使う 例えば\figno{\ref{fig:layout}}のように...
\item[ b) すぐうしろが括弧等の場合:] \verb+\figno*{\ref{fig:layout}}+を
使う 例えば「\figno*{\ref{fig:layout}}」のように...
\end{Description}
前者の「の」と図番号との間の四分空きが,後者の
場合には入(ってはな)らない.
表と写真とに対してはそれぞれ \verb+\tabno+, \verb+\photono+および
その$*$印付きが対応している.

 また英文の場合で複数の引用をする場合は
\begin{verbatim}
   \figsno{\ref{fig:1}, \ref{fig:2}}{\ref{fig:last}}
\end{verbatim}
のようにする.表は \verb+\tabsno+,写真は \verb+\photosno+である.
%
\item[ 副キャプション:] 図を2枚横に並べて副番号を振る場合には,
標準の \verb+\caption+以外に \verb+\subcaption+が使える.
引用も通常と同じで,\verb+\subcaption+直後の \verb+\label+に
副記号が含まれる.キャプションの位置は図の場合は図の下,
表の場合は表の上である.例えば
\begin{figure}[t]
  \begin{center}
   \subpicture{一枚目}
    \subcaption{最初の図}
     \label{fig:1}
  \end{center}
  \begin{minipage}[t]{.47\textwidth}
   ~
   \begin{center}
    \subpicture{田村正和}
     \subcaption{次の図}
      \label{fig:2}
   \end{center}
  \end{minipage}
  ~
  \begin{minipage}[t]{.47\textwidth}
   ~
   \begin{center}
    \subpicture{三枚目}
     \subcaption{最後の図}
      \label{fig:3}
   \end{center}
  \end{minipage}
\caption{図,あれこれ}
\label{fig:all}
\end{figure}

\renewcommand{\baselinestretch}{0.75}\small\normalsize
\begin{verbatim}
\begin{figure}[h]
  ......
    \subcaption{あれがこうだった場合の応答}
     \label{fig:1}
  ......
    \subcaption{それがあんなだった場合の応答}
     \label{fig:2}
\caption{なんだかんだの応答}
\label{fig:all}
\end{figure}
\end{verbatim}
\renewcommand{\baselinestretch}{1}\small\normalsize
のようにしてラベルを設定する.
\figno{\ref{fig:all}}がその例である.
もし \verb+\NoSubfloatCaptionHead+を宣言すると,
サブキャプションには図表番号も現われなくなる.
\tabno{\ref{tab:all}}がその例である.
\begin{table}[t]
\NoSubfloatCaptionHead
\caption{番号無しの場合の表の例}
\label{tab:all}
    \subcaption{最初の表}
     \label{tab:1}
  \begin{center}
   \subpicture{一枚目}
  \end{center}
  \begin{minipage}[t]{.47\textwidth}
   ~
     \subcaption{次の表}
      \label{tab:2}
   \begin{center}
    \subpicture{二枚目}
   \end{center}
  \end{minipage}
  ~
  \begin{minipage}[t]{.47\textwidth}
   ~
     \subcaption{最後の表}
      \label{tab:3}
   \begin{center}
    \subpicture{三枚目}
   \end{center}
  \end{minipage}
\end{table}
%
\item[ 破線:] 表の水平線に破線を使うには,
標準の \verb+\cline+の代わりに \verb+\dline+を用いる.
使い方は \verb+\cline+と同じ.
破線の長さは \verb+\dlinepattern{2pt}{4pt}+のよう(デフォルト)に
定義できる.
\vskip 5mm

\begin{minipage}[c]{.45\textwidth}
\renewcommand{\baselinestretch}{0.75}\small\normalsize
\begin{verbatim}
  \begin{tabular}{|l|l|} \hline
   A & 1st line \\ \dline{2-2}
   B & 2nd line \\ \dline{1-2}
  \end{tabular}
\end{verbatim}
\renewcommand{\baselinestretch}{1}\small\normalsize
\end{minipage}
~~
\begin{minipage}[c]{.35\textwidth}
\begin{center}
\begin{tabular}{|l|l|} \hline
 A & 1st line \\ \dline{2-2}
 B & 2nd line \\ \dline{1-2}
\end{tabular}
\end{center}
\end{minipage}
%
\end{Description}

\subsection{数式}

\begin{Description}
%
\item[ スペーシング:] 標準の{\tt eqnarray}環境のタブ部分のスペースを
小さくした.特に講演概要集等では必須かも.
%
\item[ 副番号付き複数式列挙:] 標準の{\tt eqnarray}環境と
同じ使い方で{\tt manyeqns}環境を
使う.\verb+\begin{manyeqns}+直後の \verb+\label+が
メインの式番号を,式のあとの \verb+\label+が副番号付きの
式番号を設定する.

\vskip 5mm
\begin{minipage}[c]{.5\textwidth}
\renewcommand{\baselinestretch}{.75}\small\normalsize
\begin{verbatim}
  \begin{manyeqns}
  \label{eq:thiseq}
   F & = & \int_\Gamma \sin z
    \, \d z \label{eq:thiseq1} \\
   G & = & \sum_{n=0}^{\infty}
      a_n \, t^n \label{eq:2nd}
  \end{manyeqns}
  式(\ref{eq:2nd})が二番目で
    式(\ref{eq:thiseq1})が最初の
  式だが,どちらも番号
    そのものは(\ref{eq:thiseq})である.
\end{verbatim}
\renewcommand{\baselinestretch}{1}\small\normalsize
\end{minipage}
\hfill
\begin{minipage}[c]{.4\textwidth}
\begin{manyeqns}
\label{eq:thiseq}
 F & = & \int_\Gamma \sin z \, \d z \label{eq:thiseq1} \\
 G & = & \sum_{n=0}^{\infty} a_n \, t^n \label{eq:2nd}
\end{manyeqns}
式(\ref{eq:2nd})が二番目で式(\ref{eq:thiseq1})が最初の
式だが,どちらも番号そのものは(\ref{eq:thiseq})である.
\end{minipage}
%
\item[ 分数:] 強制的に{\tt displaystyle}の分数にする \verb+\dfrac+を
使うと,文中でも$\dfrac12$のよう(\verb+\dfrac12+)になる.
また,スラッシュを用いた分数の場合は \verb+\slfrac+を
使うと$\slfrac12$のように(\verb+\slfrac12+)なる.
%
\item[ ベクトル:] 結構面倒なので,\verb+\fat+で太くなるようにした.
例えば$\fat{v}$のように(\verb+\fat{v}+)なる.
%
\item[ 「よって」と「何故ならば」:] これは標準には含まれていない.
そこで \verb+\therefore+と \verb+\because+が定義されている.
それぞれ$\therefore$, $\because$となる.
%
\item[ 行列中の大きい零:] 「楽々」にあった定義を
そのまま入れた.\verb+\bigzerol+と \verb+\bigzerou+である.
\vskip 5mm

\begin{minipage}[c]{.35\textwidth}
\renewcommand{\baselinestretch}{.75}\small\normalsize
\begin{verbatim}
  \left( \begin{array}{ccc}
   \alpha & & \bigzerou\\
   \cdots & \ddots & \cdots\\
   \bigzerol & & \beta
  \end{array} \right)
\end{verbatim}
\renewcommand{\baselinestretch}{1}\small\normalsize
\end{minipage}
\hfill
\begin{minipage}[c]{.35\textwidth}
\begin{eqnarray*}
\left( \begin{array}{ccc}
 \alpha & & \bigzerou\\
 \cdots & \ddots & \cdots\\
 \bigzerol & & \beta
\end{array} \right)
\end{eqnarray*}
\end{minipage}
%
\begin{table}
\caption{微係数表示マクロの使用例---入力間違いに注意}
\label{tab:derivative}
\renewcommand{\arraystretch}{1.7}
\begin{center}
\begin{tabular}{|l|l||l|l|}\hline
\verb+$\D{u(x,y)}{x}$+ &
$\displaystyle\D{u(x,y)}{x}$ &
\verb+$\D[4]{u(x,y)}{x}$+ &
$\displaystyle\D[4]{u(x,y)}{x}$ \\ \hline
%
\verb+$\D[4][3][y]{u(x,y)}{x}$+ &
$\displaystyle\D[4][3][y]{u(x,y)}{x}$ &
\verb+$\D[4][2][y]{u(x,y)}{x}$+ &
$\displaystyle\D[4][2][y]{u(x,y)}{x}$ \\ \hline
%
\verb+$\D[4][1][y]{u(x,y)}{x}$+ &
$\displaystyle\D[4][1][y]{u(x,y)}{x}$ &
\verb+$\D[4][4][y]{u(x,y)}{x}$+ &
$\displaystyle\D[4][4][y]{u(x,y)}{x}$ \\ \hline
%
\verb+$\D[4][0][y]{u(x,y)}{x}$+ &
$\displaystyle\D[4][0][y]{u(x,y)}{x}$ &
\verb+$\D[4][2]{u(x,y)}{x}$+ &
$\displaystyle\D[4][2]{u(x,y)}{x}$ \\ \hline
%
\verb+$\D*{u(x)}{x}$+ &
$\displaystyle\D*{u(x)}{x}$ &
\verb+$\D*[2]{u(x)}{x}$+ &
$\displaystyle\D*[2]{u(x)}{x}$ \\ \hline
\end{tabular}
\end{center}
\end{table}
%
\item[ 添え字:] 単なる指標ではなく`cr'等意味のある添え字は
ローマン体である.
簡便にできるように2種類の上下添え字を定義した.
いずれも数式モードで使用.
\vskip 1ex
\begin{minipage}[c]{.45\textwidth}
\renewcommand{\baselinestretch}{0.75}\small\normalsize
\begin{verbatim}
 \sigma\sub{cr}, \sigma\subsc{y},
  \Omega\super{max}, \Omega\supersc{a}
\end{verbatim}
\renewcommand{\baselinestretch}{1}\small\normalsize
\end{minipage}
\hfill$\to$\hfill
\begin{minipage}[c]{.35\textwidth}
$\sigma\sub{cr}$, $\sigma\subsc{y}$,
 $\Omega\super{max}$, $\Omega\supersc{a}$
\end{minipage}
\vskip 1ex
つまり,小文字のときは \verb+\sub+, \verb+\super+を用い,
大文字のときは \verb+\subsc+, \verb+\supersc+を使うが
文字そのものは小文字で指定(small capsフォントを用いている)する.
%
\item[ 積分:] 積分記号の最後に付ける$\d x$等は,実は$d x$ではない.
つまりdが通常のローマンである.そのために \verb+\d+を定義した.
これでアクセントの \verb+\d+は使えなくなっている.
\vskip 1ex
\begin{minipage}[c]{.45\textwidth}
\renewcommand{\baselinestretch}{0.75}\small\normalsize
\begin{verbatim}
\int_0^\ell f(x)\d x
\end{verbatim}
\renewcommand{\baselinestretch}{1}\small\normalsize
\end{minipage}
\hfill$\to$\hfill
\begin{minipage}[c]{.35\textwidth}
$\displaystyle\int_0^\ell f(x)\d x$
\end{minipage}
\vskip 1ex
%
\item[ 微分:] 同様に常微分のdもローマンらしい.
文中で上記の \verb+\slfrac+を用いて微係数を表示する
場合の$\slfrac{\slfracd f(x)}{\slfracd x}$は
\begin{verbatim}
     \slfrac{\slfracd f(x)}{\slfracd x}
\end{verbatim}
それ以外では上記の \verb+\d+を使用する.
%
\item[ 微係数:] 結構面倒なのでマクロを
組んだ.\tabno{\ref{tab:derivative}}参照.
\end{Description}

\subsection{文字や相互参照}

\begin{Description}
%
\item[ 丸囲み数字:] 丸で囲まれた数字で \verb+\MARU{12}+のように
して用いると\MARU{12}と出力される.
%
\item[ 相互参照のスペーシング調整:] 節\refno{\ref{sec:1}}に
示したように,アスキー文字と漢字との間には四分空きが
入らなければならないが,\verb+\ref+を用いた場合には
それが入らない.これを改善するために新しく \verb+\refno+と
いうコマンドを定義した.「\verb+節\refno{\ref{sec:1}}に+」の
ように使う.
%
\item[ 文献参照:] 土木学会規程通り,
参照する文献番号は右上肩に片括弧付きで表示する.しかし,文中に
どうしても書きたい場合もある.例えば「・・は文献2)を参照し」の
ような場合である.学会論文集ではそれを容認するとは考え難いが,
それ以外では \verb+\cite+の代りに \verb+\textcite+を使えば,
そのような出力になる.
\end{Description}

\subsection{箇条書き環境}

スペースを確保するために箇条書き
環境の上下の空きを小さくした.もし通常の設定値で
箇条書き環境を使いたい場合には,大文字で始まる{\tt Itemize},
 {\tt Enumerate}, {\tt Description}環境を用いる.

\lastpagesettings

\end{document}

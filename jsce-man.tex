%%%%%%%%%%%%%%%%%%%%%%%%%%%%%%%%%%%%%%%%%%%%%%%%%%%%%%%%%%%%%%%%%%%%%%
%                                                                    %
% 土木学会論文集マニュアル                                           %
% jsce.sty                                                           %
%                                                                    %
%%%%%%%%%%%%%%%%%%%%%%%%%%%%%%%%%%%%%%%%%%%%%%%%%%%%%%%%%%%%%%%%%%%%%%
\typeout{^^J*************************************************}
\typeout{** 土木学会論文集スタイルファイル用マニュアル  **}
\typeout{** アスキーpLaTeXで2回コンパイルしてください.**}
\typeout{** マニュアル作成にはjsce.clsを使います.      **}
\typeout{*************************************************^^J}
%

\documentclass[onecolumn]{jsce}  % for LaTeX2e users 
%
\usepackage[varg]{txfonts}
%
\finalversion
\receivedate{2014年4月4日}
\title{土木学会論文集用の原稿作成マニュアル}
\endtitle{A Manual of JSCE Style Files for \LaTeX}
\author{土木学会論文集編集委員会\thanks{公益社団法人 土木学会
(〒160-0004東京都新宿区四谷一丁目外濠公園内)\email{03-3355-3441}}}
\endauthor{JSCE Committee}
\abstract{土木学会論文集の原稿を作成するためのマニュアル.同時に,
応用力学と構造工学の論文集のスタイルファイル利用のマニュアルでもある.}
\keywords{class file, style files, \LaTeXe, p\LaTeX, J\LaTeX,
 Japan Society of Civil Engineers Journals, dvi2pdf}
%
\unitlength=1mm
\def\subpicture#1{\begin{picture}(50,7)(0,0)
\put(0,0){\framebox(50,7){#1}}\end{picture}}
\def\BibTeX{{\rm B\kern-.05em{\sc i\kern-.025em b}\kern-.08em
    T\kern-.1667em\lower.7ex\hbox{E}\kern-.125emX}}
\def\JBibTeX{\leavevmode\lower .6ex\hbox{J}\kern-0.15em\BibTeX}
% \pagestyle{plain}
%
\begin{document}
\maketitle

\renewcommand{\baselinestretch}{0.95}\small
\begin{quote}
\tableofcontents
\end{quote}
\renewcommand{\baselinestretch}{1}\small\normalsize

\section{\LaTeX\ のバージョンとお願い}
\label{sec:1}

\subsection{バージョンとサポート}

このクラスファイル`{\tt jsce.cls}'は\LaTeXe\ 用
である.\LaTeX\ 209をどうしても
使いたい場合には,それ専用のスタイルファイルを使用して欲しいが,
そのメインテナンスは最近はしていないので注意して欲しい.
またこのマニュアルと土木学会論文集原稿は,
アスキーのp\LaTeX\ でしかコンパイルできない.
角藤版の配布アーカイブの中なら{\tt platex}によるコンパイル用である.
本バージョンからNTT J\TeX\ 用は配布していないので注意する.
ただし,応用力学論文集(`{\tt applm-2e.sty}')と
構造工学論文集(`{\tt st-symp.sty}')の
スタイルファイルはNTT J\TeX\ で(開発したのでそれで)も使える.
この三つの論文集に対してはほぼ同じ設定でクラス・スタイルファイル類が
作られているので,このマニュアルがほぼそのまま適用できる.

\subsection{ソース入力上の注意事項}

さて,組版規則のほとんどは\TeX\ が対処してくれているので,
著者はレイアウト等のことは一切考えずに,論文の論理構成だけに
精神を集中させることができる.例えば,
科学研究費申請も\LaTeX\ 化が進んでから苦痛がほとんど無くなった人は
多いだろうと想像できる.
つまり,例えば \verb+\hspace+や \verb+\baselineskip+のような
レイアウト調整のコマンドは,図表環境等を除いた本文部分では
一切使う必要は無いし,使わないようにして欲しい.
しかし,ソース入力時に著者自身が注意しないといけない部分があるので,
ここ以降にいくつかまとめた.\LaTeX\ を普段使っている方々も,
論文の執筆前に必ずこのマニュアルを一通り読んで欲しい.
また,クラスファイルやスタイルファイルを改変しないで
いただきたい.\LaTeX\ に
よる投稿を某社のワードプロセサによる投稿と同じ扱いにしてもらう
ために,編集委員の一部ボランティアが多大なる努力を尽くしてくださった.
そのご努力を無駄にしないためにも,ファイル類の改変はしないで
いただきたい.不具合が見つかった場合には学会の担当者に報告していただければ,
またボランティアによる改訂は可能であろうと予想される.

まず大原則を書いておくと以下のような項目になろうかと思う.
\begin{enumerate}
\item 著者がレイアウトを制御するようなことをしない.
\begin{enumerate}
\item \verb+\vskip+や \verb+\vspace+あるいは \verb+\baseline+関連の
利用は,図表のフロート部以外では使わない.
\item 例えば改行しない半角スペース \verb+~+を使って式のレイアウト等をしない.
\item 式中も,誤解が無いように \verb+\quad+や
タブ \verb+&+ のみを用いて式を並べ,\verb+~+等でレイアウトしない.
\end{enumerate}
\item 本文中の半角スペースは,改行不可の特殊なスペース \verb+~+を
使わずに普通のスペースを使う.
\item 全角・半角を問わず,余計なスペースを著者が挿入しない.
\item 標準的な\LaTeX\ の使い方に留める.
\end{enumerate}
さて,p\LaTeX\ の特徴である(改行位置についてはNTT J\LaTeX\ では
気にしないでいい)が,
よく見られる代表的な不具合を含んだソースの2行の例を左端に示す.
ここの \verb*+ + は半角スペースである.
\smallskip

\noindent\mbox{}\hfill
\begin{minipage}[t]{.28\textwidth}
\begin{verbatim*}
・・・は bending rigidity
が一様では・・・
\end{verbatim*}
\end{minipage}
\hfill
\begin{minipage}[t]{.28\textwidth}
\begin{verbatim*}
・・・はbending
   rigidityが一様では・・・
\end{verbatim*}
\end{minipage}
\hfill
\begin{minipage}[t]{.28\textwidth}
\begin{verbatim*}
・・・はbending rigidityが
一様では・・・
\end{verbatim*}
\end{minipage}
\hfill\mbox{}

\medskip
\noindent
良くない箇所は,一つは「は」と`b'の間に半角スペースがあること,
もう一つは`rigidity'の半角で改行されて次の行頭が全角文字であることである.
全角と半角の文字間には適切なスペースが自動的に挿入されるので,さらに
半角スペースを足すと間延びしてしまう.
したがって`b'の前のスペースは削除する.
また半角文字の改行位置には(英文執筆を考えてみれば明らかなように)
自動的に半角文字が挿入されてしまうため,次の行頭の全角文字との
間に余計なスペースが挿入されてしまう.
このような場合には,`rigidity'の`y'の直後に`\%'を付すか,
上の右の二つの例のように改行位置を適切な場所に変更する必要がある.

\subsection{\LaTeX\ コマンドを利用して論理構成を分かり易く組む}

また,節の構成は論理性を明確化するのに役立つことも念頭に置いて欲しい.
例えば
\medskip

\noindent
\mbox{}\hfill
\begin{minipage}[t]{.42\textwidth}
\begin{verbatim}
\section{一つの主な節}
\subsection{その中の最初の言葉}
これが,それが,あれが,どれが・・・
\subsection{それに基づいた次のステップ}
こんな,そんな,あんな,どんな・・・
\section{次の主な節}
\end{verbatim}
\end{minipage}
\hfill
\begin{minipage}[t]{.42\textwidth}
\begin{verbatim}
\section{一つの主な節}
一種の予告編のような,最初の言葉・・・
\subsection{それに基づいた次のステップ}
こんな,そんな,あんな,どんな・・・
\section{次の主な節}
これが,それが,あれが,どれが・・・
\end{verbatim}
\end{minipage}
\hfill\mbox{}
\medskip

\noindent
の二つを眺めたとき,左の方が明確だろう.それに比べて
右の論理は不明瞭だ.まず一つの節に小節が一つしかないから,
なぜこの小節が必要なのかもわからない.
さらに,その小節よりも前に予告編のような
題目の無い節が置いてあり,この節の論理が不明確になっている.
同じことは,小節の中の \verb+\subsubsection+に対しても当てはまる.
節題目が何のために必要なのかを考えながら,節の構成をまず
分かり易いものにするべきだろう.

\section{基本情報の設定と本文中の注意および最終ページの調整}
\label{sec:2}

\subsection{題目や著者・概要}

学会公式版になったときに,スタイルファイルではなくクラスファイルになったので,
その`{\tt jsce.cls}'を使えば,オプション無しで2段組になる.
ただし,それ以前の非公式版や准公式版で使えていたマクロ等は
そのまま引き継いでいる.
題目や著者・概要等はプリアンブル部で次のように設定する.

\renewcommand{\baselinestretch}{0.75}\small\normalsize
\begin{verbatim}
     \documentclass{jsce}
     % \inenglish         % 英語で書く場合にはアンコメント
     \finalversion        % ヘッダ不要
     %
     \title{メインのタイトル:和文なら日本語}
     \endtitle{別言語のタイトル:和文なら英語}
     \author{著者1 \thanks{正会員 学位 職 所属(住所)}・
             著者2 \thanks{....\email{your_name@foo.ac.jp}}} % 最低一人は Email 必要
     \endauthor{別言語の著者リスト}
     \abstract{概要....空行可}   % 空行入れれば段落
     \keywords{キーワード}
     \endabstract{別言語の概要...空行可}
     % \titlepagecontrol{1} % メインタイトル部分と本文との間隔調整(通常不要)
     \receivedate{2012. 10. 10} % 受理された年月日.英文なら October 10, 2012
     % \receivedate を省略するとコンパイルした日付が入る
     \begin{document}
     \maketitle
\end{verbatim}
\renewcommand{\baselinestretch}{1}\small\normalsize
%
論文題目は \verb+\title+で定義する.
他言語題目が \verb+\endtitle+であり,こちらは論文の最終ページに
配置される.和文の場合の著者情報は \verb+\author+の中に,中黒「・」を
区切り文字にして列挙する.個々の身分や連絡先は \verb+\thanks+の
中で定義しておく.
著者のうち最低でも一人はメイルアドレスを \verb+\email+で
定義しておく必要があり,\verb+\thanks+の中で
設定する.\verb+\endauthor+が英文の著者の列挙で,
通常の英語を用いた列挙規則で並べる.\verb+\abstract+に和文概要を
記すが,空行を入れて複数の段落も可能である.そのとき段落始めの
字下げは入れないこと.\verb+\keywords+に
半角文字だけでキーワードをコンマ区切りで羅列する.`{\tt and}'は
不要である.\verb+\endabstract+が英文概要で複数の段落が許容され,
論文の最終ページに配置される.
英語で執筆する場合には \verb+\inenglish+をアンコメントする.
最後に提出日を \verb+\receivedate+ (receive[d-d]ate)に設定しておく.
本文最初の \verb+\maketitle+で1段組の部分が組まれ,
本文はそれ以降に書き込むことになる.\verb+\titlepagecontrol+は
通常は設定する必要が無いが,
特に最初のページで段落間に異様な空行が現れた際に使う調整コマンドである.
この調整の仕方の詳細は第\ref{sec:qanda}節のA.\ref{item:titlepagecontrol}に
書いておいた.
応用力学論文集と構造工学論文集の場合もほぼ同様なので
サンプルで推測して欲しい.

\subsection{本文を書くときの注意}
\label{sec:2-1}

\subsubsection{文章の書き方}
\label{sec:2-1-1}

老婆心ながら本質的な部分についての情報を,学生さん対応として書いておく.
科学論文は誤解無く読み手に情報を伝えるものだという共通認識が
あるものの,査読時に苦労しておられる方も多いと思われる.
このマニュアルもさほど推敲されているとはお感じにならないかもしれない.
一つわかり易い解説書\cite{honda}がある.最初の120ページくらいを読むだけで,
文章のわかり易さが格段に増すことが期待できる.

\subsubsection{節建て}

節は必ず \verb+\section+等のコマンドで設定すること.
階層は \verb+\subsubsection+までで留め,\verb+\paragraph+等は
謝辞以外では原則使わない.できるだけ標準的な\LaTeX\ で執筆すること.
特殊なマクロ等や標準ではないスタイルファイル類は可能な限り
使わない方が望ましい.
そういった非標準のマクロの中には,レイアウトに影響を与える
ような設定が存在する可能性があるからである.

\subsubsection{区切り文字}

横書きの組版規則に関連して
土木学会規程では,区切り文字には全角のコンマ`,'とピリオド`.'を用いる.
ただ文部科学省は`,'と`。'を用いるように指導しているので,
教科書等ではそうなっているのが多いので注意する.
句読点`、'`。'や半角のコンマとピリオドは和文本文中では使わないこと.
ただし,記号のような半角文字を羅列する場合には半角のコンマを
使う方が読み易くなる.このときコンマの次の半角のスペースを忘れないこと.

\subsubsection{全角と半角文字の取り合い}

昨今のワードプロセサユーザーのほとんどが
間違っているが,{\bf 英数字はすべて半角}にしなければならない.
さて組版規則では,全角文字と半角文字の間には四分空きを入れるように
なっているが,
日本語\TeX\ がそれを自動的に処理してくれるので,著者が{\bf 余分な空白を
入れたりしない}ようにすること.
特に,全角スペースで位置合わせをするとかえって読み難い文章になる
可能性もあるので,表中等も含めて本文中では用いないようにする.

一方,{\bf 単位と数値の間にも四分空きが必要}なのだが,これを
忘れる人が多い.
例えば「5キロニュートン・パァ・平方ミリ」を表すには「\verb+5\,kN/mm$^2$+」と
すると「5\,kN/mm$^2$」のように表示される.\verb+\,+を入れないと
「5kN/mm$^2$」となって読み難くなるので注意する.
ただし,温度「21\degC」とパーセント「34.5\%」の前は空けないそうだ.
なお「キロ」は小文字であり,単位は斜体にはしない.

また,例えば括弧やコンマ等の区切り文字について,全角の間なら全角を,
半角の間なら半角のスペースと区切り文字を使う方が読み易くなる.
ただし参考文献リストにおいてのみ,区切り文字を半角の
スペース付きの区切り文字にする方がいい.
これは幅が狭い段の中に文字を配置する場合に,半角の方が\TeX\ が
制御し易いからである.
リストは情報であって読み物部分ではないので,半角区切り文字と
半角スペースで十分である.

\subsubsection{図表の引用}

引用には\figno{\ref{fig:layout}}のようにゴシック体を用いるので注意する.
ただし,構造工学論文集では明朝体のままである.
クラスファイル等で定義した \verb+\figno{\ref{label}}+等を用いれば,
自動的にそれぞれでゴシックか明朝になる.

\subsubsection{参考文献リスト}

上述のように区切り文字は半角のスペース付きの半角で構わない.
このマニュアルのソースファイルの文献リストを参考にして欲しいが,
著書タイトルや論文集名等のイタリック体はコンマまでを含めること.
よく忘れられているのは行末の発行年の後の半角コンマである.
そのとき発行月は原則付さない.
発行月を付す場合には,すべての文献に付して統一をとる.
また号毎に1からページが振られる地盤工学会のS \& Fのような場合を除き,
巻号の号は付けないことを原則としていい.
文献の著者は中黒ではなくコンマで区切り,英語の著者の名前は
ピリオド付きのイニシャルだけにし,姓を先に書く.
著者一覧の最後はコロンで区切る.
巻号には「巻」「Vol.」や「号」「No.」を省略せず,
ページの範囲指定の「pp.」あるいは一頁指定の場合も「p.」を省略しない.



\subsection{最終ページの調整}

\begin{figure}
\begin{center}
\setlength{\unitlength}{1mm}
\begin{picture}(110,65)(0,0)
\put(5,5){\thicklines\framebox(40,55){}}
\put(60,5){\thicklines\framebox(40,55){}}
\put(10,10){\dashbox{0.3}(13,45){}}
\put(27,38){\dashbox{0.3}(13,17){}}
\put(65,24){\dashbox{0.3}(13,31){}}
\put(82,24){\dashbox{0.3}(13,31){}}
\put(65,15){\dashbox{0.5}(30,7){他言語タイトル等}}
\put(5,60){\makebox(40,5){最終頁}}
\put(60,60){\makebox(40,5){最終頁}}
\put(5,0){\makebox(40,5){調整前}}
\put(60,0){\makebox(40,5){調整後}}
\put(48,35){\vector(1,0){9}}
\put(30,20){\vector(0,1){17}}
\put(30,20){\vector(0,-1){10}}
\put(66,15){\vector(0,1){7}}
\put(66,15){\vector(0,-1){5}}
\put(30,20){\makebox(15,5){$x$ cm}}
\put(67,5){\makebox(25,10)[l]{$x/2$ cm}}
\put(82,24){\makebox(13,5)[r]{\footnotesize (受理)}}
\put(27,38){\makebox(13,5)[r]{\footnotesize (受理)}}
\put(66,50){\makebox(13,5)[l]{\footnotesize {\tt \symbol{'134}lastpage-}}}
\put(67,47){\makebox(13,5)[l]{\footnotesize {\tt control\{x/2\}}}}
\put(28,50){\makebox(13,5)[l]{\footnotesize 4. 結論}}
\put(83,36){\makebox(13,5)[l]{\footnotesize 4. 結論}}
\end{picture}
\end{center}
\caption{最終頁レイアウト調整の概略}
\label{fig:layout}
\end{figure}

論文の最後の頁には2段のバランスを取った上で,
他言語によるタイトル他が配置される.
そのためにはまず1回コンパイルし,\figno{\ref{fig:layout}}の
左図のように右段の下空白部の高さ`{\tt x}\,cm'を測定する.
そうした上で左段の上方にくる文章のどこかに`{\tt x}'の半分の値
(例としてそれが8\,cmになったとして)で
\begin{verbatim}
     \lastpagecontrol{8cm}
\end{verbatim}
といった調整コマンドを挿入する.
これで,この頁の下に他言語タイトル等が出力される.
この高さをどのくらいにすると読み易いかについては,trial and errorで
調整する.
もし本文との間にさらに空白が必要な場合には
\begin{verbatim}
     \lastpagecontrol[5mm]{8cm}
\end{verbatim}
のように追加スペース(この例は5\,mm)を指定する.
この追加分は他言語タイトル等の箱と本文との間のスペースである.
この空白部分にタイトル等が入りきらない場合には,
次のコマンドで自動的に次頁の頭に出力される.

最後の最後は受理年月日を出力するために,本文の最後の行に
\begin{verbatim}
     \lastpagesettings
\end{verbatim}
というコマンドを挿入する.
ただし,もし上の \verb+\lastpagecontrol+で他言語タイトル等が
入りきらない場合には,次の頁の頭に他言語タイトル等が設定されると書いたが,
このとき,タイトル等よりも上方の空白を
\begin{verbatim}
     \lastpagesettings[3cm]
\end{verbatim}
のようにも指定できる.




\section{標準以外に追加した新しいコマンド・環境等}

\subsection{フロートや表}

\begin{Description}
\parindent=1zw
%
\item[写真環境:] \verb+\begin{photo}+で始まる環境で,{\tt figure},
 {\tt table}環境と同じフロートである.
%
\item[図表番号の引用:] 本文中で図表番号を引用する場合には,
例えば図なら \verb+\figno{\ref{fig:setting}}+等が使える.
ただし漢字とのスペースの関係で,次のように使い分ける.
\begin{Description}
\item[a) すぐうしろが漢字の場合:] \verb+\figno{\ref{fig:layout}}+を
使うと,例えば\figno{\ref{fig:layout}}のようになる.
\item[b) すぐうしろが括弧等の場合:] \verb+\figno*{\ref{fig:layout}}+を
使うと,例えば「\figno*{\ref{fig:layout}}」のようになる.
\end{Description}
前者の「の」と図番号との間の四分空きが,後者の
場合には入(ってはな)らない.後者で
アスタリスク無しだと「\figno{\ref{fig:layout}}」のように
空白が入ってしまう.
表と写真とに対してはそれぞれ \verb+\tabno+, \verb+\photono+および
そのアスタリスク付きが対応している.
和文で二つの引用は,\verb+\ref+を並べて \verb+\tabno{\ref{1}, \ref{2}}+と
する(間に半角スペース必要)だけでいいが,実は
複数の図表を一遍に引用するのは読者には不親切である.
一つ一つを順に引用すべきであるし,比較するときも「○○の
場合の{\bf \figurename1}と△△の場合の{\bf \figurename2}を
比較して明らかなように,」とした方が,
着目すべき違いが明らかになって理解し易い.
よく「{\bf \figurename 2〜8}に結果を示したが,」という『予告編』を
書く学生さんがいるが,こんな文章は無くても,
読者の理解には全く違いが無いのは明らかだろう.

なお英文の場合,複数の引用をする場合は
\begin{verbatim}
   \figsno{\ref{fig:1}, \ref{fig:2}}{\ref{fig:last}}
\end{verbatim}
のようにすると{\bf Figs.1, 2 and 3}のようになる.
表は \verb+\tabsno+,写真は \verb+\photosno+である.
%
\begin{figure}
  \begin{center}
   \subpicture{1枚目}
  \end{center}
   \vspace*{-5mm}
    \subcaption{最初の図}
     \label{fig:1}
  \vspace*{-5mm}
  \begin{minipage}[t]{.47\textwidth}
   ~
   \begin{center}
    \subpicture{2枚目}
   \end{center}
   \vspace*{-5mm}
     \subcaption{次の図}
      \label{fig:2}
  \end{minipage}
  ~
  \begin{minipage}[t]{.47\textwidth}
   ~
   \begin{center}
    \subpicture{3枚目}
   \end{center}
   \vspace*{-5mm}
     \subcaption{最後の図}
      \label{fig:3}
  \end{minipage}
 \vspace*{-4mm}
\caption{図,あれこれ}
\label{fig:all}
\end{figure}
%
\item[副キャプション:] 図を2枚横に並べて副番号を振る場合には,
標準の \verb+\caption+以外に \verb+\subcaption+が使える.
引用も通常と同じで,\verb+\subcaption+直後の \verb+\label+に
副記号が含まれる.キャプションの位置は図の場合は図の下,
表の場合は表の上である.
例えば

\renewcommand{\baselinestretch}{0.75}\small\normalsize
\begin{verbatim}
\begin{figure}
  ......
    \subcaption{あれがこうだった場合の応答}
     \label{fig:1}
  ......
    \subcaption{それがあんなだった場合の応答}
     \label{fig:2}
\caption{なんだかんだの応答}
\label{fig:all}
\end{figure}
\end{verbatim}
\renewcommand{\baselinestretch}{1}\small\normalsize
のようにしてラベルを設定する.
\figno{\ref{fig:all}}がその例である.
もし \verb+\NoSubfloatCaptionHead+を宣言すると,
サブキャプションには図表番号も現われなくなる.
\tabno{\ref{tab:all}}がその例である.

\begin{table}
\NoSubfloatCaptionHead
\caption{番号無しの場合の表の例}
\label{tab:all}
\vspace*{-5mm}
  \subcaption{最初の表}
     \label{tab:1}
  \vspace*{2mm}
  \begin{center}
   \subpicture{一つ目}
  \end{center}
  \begin{minipage}[t]{.47\textwidth}
 \vspace*{-8mm}
     \subcaption{次の表}
      \label{tab:2}
  \vspace*{2mm}
   \begin{center}
    \subpicture{二つ目}
   \end{center}
  \end{minipage}
  \begin{minipage}[t]{.47\textwidth}
 \vspace*{-8mm}
     \subcaption{最後の表}
      \label{tab:3}
  \vspace*{2mm}
   \begin{center}
    \subpicture{三つ目}
   \end{center}
  \end{minipage}
\end{table}
%
\item[破線:] 表の水平線に破線を使うには,
標準の \verb+\cline+の代わりに \verb+\dline+を用いる.
使い方は \verb+\cline+と同じ.
破線の長さは \verb+\dlinepattern{2pt}{4pt}+のよう(デフォルト)に
定義できる.
% \medskip

\noindent
\mbox{}\hfill
\begin{minipage}[c]{.45\textwidth}
\renewcommand{\baselinestretch}{0.75}\small\normalsize
\begin{verbatim}
  \begin{tabular}{|l|l|} \hline
   A & 1st line \\ \dline{2-2}
   B & 2nd line \\ \dline{1-2}
  \end{tabular}
\end{verbatim}
\renewcommand{\baselinestretch}{1}\small\normalsize
\end{minipage}
~~
\begin{minipage}[c]{.35\textwidth}
\begin{center}
\begin{tabular}{|l|l|} \hline
 A & 1st line \\ \dline{2-2}
 B & 2nd line \\ \dline{1-2}
\end{tabular}
\end{center}
\end{minipage}
\hfill\mbox{}
%
\end{Description}

\subsection{数式}

\begin{Description}
\parindent=1zw
%
\item[スペーシング:] 標準の`{\tt eqnarray}'環境のタブ部分のスペースを
小さくした.その他,式の上下の文章との間のスペースも小さくしてある.
したがって,AMSのスタイルファイル等を用いた場合の式環境では若干スペースが
大きくなるかもしれないが,スペーシングのコマンドを著者が
局所的に用いてそれを無理に詰めるのはあまり望ましくない.
本来は,AMSのスタイルファイルそのものをhackして,そういったスペースを
微調整すべきである.どうしてもAMSを必要とする著者でボランティアをして
いただけるなら,是非とも調整したものを公開して欲しい.
不具合が無ければ,学会のクラスファイルと同梱して配布することも可能で
あろうから,編集委員会に遠慮なくご提案いただきたい.
%
\item[副記号付き複数式列挙:] 標準の`{\tt eqnarray}'環境と
同じ使い方で`{\tt manyeqns}'環境を
使う.\verb+\begin{manyeqns}+直後の \verb+\label+が
メインの式番号を,式のあとの \verb+\label+が副記号付きの
式番号を設定する.
\medskip

\noindent
\mbox{}\hfill
\begin{minipage}[c]{.45\textwidth}
\renewcommand{\baselinestretch}{.75}\small\normalsize
\begin{verbatim}
  \begin{manyeqns}
  \label{eq:theseeqs}
   F & = & \int_\Gamma \sin z
    \, \d z \label{eq:thiseq} \\
   G & = & \sum_{n=0}^{\infty}
      a_n \, t^n \label{eq:thateq}
  \end{manyeqns}
  式(\ref{eq:thateq})が二番目で
    式(\ref{eq:thiseq})が最初の
  式だが,両方の引用は
    式(\ref{eq:theseeqs})とする.
\end{verbatim}
\renewcommand{\baselinestretch}{1}\small\normalsize
\end{minipage}
~~~
\begin{minipage}[c]{.4\textwidth}
\begin{manyeqns}
 \label{eq:theseeqs}
   F & = & \int_\Gamma \sin z
    \, \d z \label{eq:thiseq} \\
   G & = & \sum_{n=0}^{\infty}
      a_n \, t^n \label{eq:thateq}
\end{manyeqns}
式(\ref{eq:thateq})が二番目で
式(\ref{eq:thiseq})が最初の
式だが,両方の引用は
式(\ref{eq:theseeqs})とする.
\end{minipage}
\hfill\mbox{}
\medskip

\noindent
ラベルの配置場所を間違えるとおかしくなるので注意して欲しい.

同じように副記号を付ける複数の式だが,1行で済む場合の環境も
定義しておこう.
上の`{\tt manyeqns}'環境に似せて`{\tt twoeqns}'環境としてみた.
\medskip

\noindent
\mbox{}\hfill
\begin{minipage}[c]{.45\textwidth}
\renewcommand{\baselinestretch}{.75}\small\normalsize
\begin{verbatim}
\begin{twoeqns}
\label{eq:two}
\EQab f(x)\equiv
 \sum_{n=1}^\infty a_n\,g_n(x),
           \quad \label{eq:two-a}
\EQab g=\sum b_n \cdots \label{eq:two-b}
\end{twoeqns}
式(\ref{eq:two-b})が二番目で,
両方の引用は式(\ref{eq:two})とする.
\end{verbatim}
\renewcommand{\baselinestretch}{1}\small\normalsize
\end{minipage}
~~~
\begin{minipage}[c]{.4\textwidth}
\begin{twoeqns}
\label{eq:two}
\EQab f(x)\equiv\sum_{n=1}^\infty a_n\,g_n(x),
 \quad \label{eq:two-a}
\EQab g=\sum b_n \cdots \label{eq:two-b}
\end{twoeqns}
式(\ref{eq:two-b})が二番目で,
両方の引用は式(\ref{eq:two})とする.
\end{minipage}
\hfill\mbox{}
\medskip

\noindent
これもラベルの配置場所を間違えるとおかしくなるので注意して欲しい.
式は1段に入るなら二つ以上でも構わないが,各式の頭に \verb+\EQab+を
忘れずに付すこと.

さらに,上の二つの統合のようなもので,副記号を付ける複数の式を
複数行に並べる場合の環境も定義しておこう.
\medskip

\noindent
\mbox{}\hfill
\begin{minipage}[c]{.45\textwidth}
\renewcommand{\baselinestretch}{.75}\small\normalsize
\begin{verbatim}
\begin{shorteqns}
\label{s0}
a&=&b, \quad \label{s1} \EQsep
  c=d, \label{s2} \\
e&=&f, \quad \label{s3} \EQsep
  g=h, \quad \label{s4} \EQsep
  i=j, \label{s5} \nonumber \\
&& k=l, \quad \label{s6} \EQsep
  m=n \label{s7}
\end{shorteqns}
全体が式(\ref{s0})で3番目が式(\ref{s3}).
\end{verbatim}
\renewcommand{\baselinestretch}{1}\small\normalsize
\end{minipage}
~~~
\begin{minipage}[c]{.4\textwidth}
\begin{shorteqns}
\label{s0}
a&=&b, \quad \label{s1} \EQsep
  c=d, \label{s2} \\
e&=&f, \quad \label{s3} \EQsep
  g=h, \quad \label{s4} \EQsep
  i=j, \label{s5} \nonumber \\
&& k=l, \quad \label{s6} \EQsep
  m=n \label{s7}
\end{shorteqns}
全体が式(\ref{s0})で3番目が式(\ref{s3}).
\end{minipage}
\hfill\mbox{}
\medskip

\noindent
同じ行の式と式の間には区切りのために \verb+\EQsep+を
付すこと.またその趣旨を考え \verb+\nonumber+は使えないようになっている.
%
\item[分数:] 強制的に`{\tt displaystyle}'の分数にする \verb+\dfrac+を
使うと,文中でも$\dfrac12$のよう(\verb+\dfrac12+)になる.
また,スラッシュを用いた分数の場合は \verb+\slfrac+を
使うと$\slfrac12$のように(\verb+\slfrac12+)なる.
%
\item[ベクトル:] 結構面倒なので,\verb+\fat+で太くなるようにした.
例えば$\fat{v}$のように(\verb+\fat{v}+)なる.
%
\item[「よって」と「何故ならば」:] これは標準には含まれていない.
そこで \verb+\therefore+と \verb+\because+が定義されている.
それぞれ$\therefore$と$\because$となる.
%
\item[行列中の大きい零:] 著名な著書「楽々\cite{rakuraku}」にあった定義を
そのまま入れた.\verb+\bigzerol+と \verb+\bigzerou+である.
\medskip

\noindent
\mbox{}\hfill
\begin{minipage}[c]{.35\textwidth}
\renewcommand{\baselinestretch}{.75}\small\normalsize
\begin{verbatim}
  \left( \begin{array}{ccc}
   \alpha & & \bigzerou\\
   \cdots & \ddots & \cdots\\
   \bigzerol & & \beta
  \end{array} \right)
\end{verbatim}
\renewcommand{\baselinestretch}{1}\small\normalsize
\end{minipage}
~~~%\hfill
\begin{minipage}[c]{.35\textwidth}
\begin{eqnarray*}
\left( \begin{array}{ccc}
 \alpha & & \bigzerou\\
 \cdots & \ddots & \cdots\\
 \bigzerol & & \beta
\end{array} \right)
\end{eqnarray*}
\end{minipage}
\hfill\mbox{}
%
\item[添え字:] 単なる指標ではなく`cr'等意味のある添え字は
ローマン体にする.
簡便にできるように2種類の上下添え字を定義した.
いずれも数式モードで使用.
\medskip

\noindent
\mbox{}\hfill
\begin{minipage}[c]{.45\textwidth}
\renewcommand{\baselinestretch}{0.75}\small\normalsize
\begin{verbatim}
 \sigma\sub{cr}, \sigma\subsc{y},
  \Omega\super{max}, \Omega\supersc{mIn}
\end{verbatim}
\renewcommand{\baselinestretch}{1}\small\normalsize
\end{minipage}
\hfill$\to$\hfill
\begin{minipage}[c]{.3\textwidth}
$\sigma\sub{cr}$, $\sigma\subsc{y}$,
 $\Omega\super{max}$, $\Omega\supersc{mIn}$
\end{minipage}
\hfill\mbox{}
\medskip

\noindent
つまり,小文字のときは \verb+\sub+, \verb+\super+を用い,
大文字のときは \verb+\subsc+, \verb+\supersc+を使うが
文字そのものはsmall capsフォントの小文字(大文字と同じ字体)を使用している.
%
\begin{table}
\caption{微係数表示マクロの使用例---入力間違いに注意}
\label{tab:derivative}
\renewcommand{\arraystretch}{1.7}
\begin{center}
\begin{tabular}{|l|l||l|l|}\hline
\verb+$\D{u(x,y)}{x}$+ &
$\displaystyle\D{u(x,y)}{x}$ &
\verb+$\D[4]{u(x,y)}{x}$+ &
$\displaystyle\D[4]{u(x,y)}{x}$ \\ \hline
%
\verb+$\D[4][3][y]{u(x,y)}{x}$+ &
$\displaystyle\D[4][3][y]{u(x,y)}{x}$ &
\verb+$\D[4][2][y]{u(x,y)}{x}$+ &
$\displaystyle\D[4][2][y]{u(x,y)}{x}$ \\ \hline
%
\verb+$\D[4][1][y]{u(x,y)}{x}$+ &
$\displaystyle\D[4][1][y]{u(x,y)}{x}$ &
\verb+$\D[4][4][y]{u(x,y)}{x}$+ &
$\displaystyle\D[4][4][y]{u(x,y)}{x}$ \\ \hline
%
\verb+$\D[4][0][y]{u(x,y)}{x}$+ &
$\displaystyle\D[4][0][y]{u(x,y)}{x}$ &
\verb+$\D[4][2]{u(x,y)}{x}$+ &
$\displaystyle\D[4][2]{u(x,y)}{x}$ \\ \hline
%
\verb+$\D*{u(x)}{x}$+ &
$\displaystyle\D*{u(x)}{x}$ &
\verb+$\D*[2]{u(x)}{x}$+ &
$\displaystyle\D*[2]{u(x)}{x}$ \\ \hline
\end{tabular}
\end{center}
\end{table}
%
\item[積分:] 積分記号の最後に付ける`$\mbox{d} x$'は,
実は`$d x$'ではない.
つまり`d'が通常のローマンである.そのために \verb+\d+を定義した.`d'の
直前に薄いスペースが挿入してある.
これでアクセントの \verb+\d+は使えなくなっているので注意する.
また,応用力学論文集のように,pdfファイルに
著者情報等を埋め込む場合には,そのための
スタイルファイル`{\tt hyperref.sty}'が \verb+\d+を
アクセントとして再定義してしまうので,元に戻す必要がある.
% \medskip

\noindent
\mbox{}\hfill
\begin{minipage}[c]{.3\textwidth}
\renewcommand{\baselinestretch}{0.75}\small\normalsize
\begin{verbatim}
\int_0^\ell f(x)\d x
\end{verbatim}
\renewcommand{\baselinestretch}{1}\small\normalsize
\end{minipage}
\hfill$\to$\hfill
\begin{minipage}[c]{.3\textwidth}
$\displaystyle\int_0^\ell f(x)\d x$
\end{minipage}
\hfill\mbox{}
\medskip
%
\item[微分:] 同様に常微分の`d'もローマンらしい.
文中で上記の \verb+\slfrac+を用いて微係数を表示する
場合の$\slfrac{\slfracd f(x)}{\slfracd x}$は
\begin{verbatim}
     \slfrac{\slfracd f(x)}{\slfracd x}
\end{verbatim}
それ以外では上記の \verb+\d+(直前に薄いスペースが入る)
か \verb+\mbox{d}+(ローマンの`d'のみ)を使用する.
%
\item[微係数:] 結構面倒なのでマクロを
組んだ.\tabno{\ref{tab:derivative}}参照.
果たして便利かどうかはわからないが.
\end{Description}

\subsection{文字や相互参照}
\label{sec:koko}

\begin{Description}
%
\item[丸囲み数字:] 丸で囲まれた数字で \verb+\MARU{12}+のように
して用いると\MARU{12}と出力される.使わない方がいい.
%
\item[摂氏の温度:] 漢字の「℃」は英文だけの論文では使えない(pdfに
したときに漢字フォントを埋め込むことになる)ので,
とりあえず一つのアイデアとして \verb+\degC+を定義しておいた.
例えば \verb+3\degC+で「3\degC」のように出力される.
%
\item[節(章)番号の引用:] 節(章)番号にピリオドが付いているので,
そのまま \verb+\ref{}+で引用するとそのピリオドも付いてしまう.
そのため,ピリオド無しの節番号だけを \verb+\Ref{}+で引用できるようにした.
例えば \verb+\ref{}+を使うと「第\ref{sec:2}節」となるが,\verb+\Ref{}+を
使うと「第\Ref{sec:2}節」のようにピリオド無しにできる.
ただし,\verb+\subsection+以下に \verb+\Ref{}+を使うとエラーに
なるので \verb+\ref{}+を用いること.
ついでにNTT J\LaTeX\ を使っている場合には,
この小節以下の引用のとき,\verb+\ref{}+の後ろに四分空きを
著者が \verb+\,+で「\verb+第\ref{sec:2-1}\,節+」のように挿入して欲しい.
%
% 効き目が無い?????
% \item[相互参照のスペーシング調整:] 節\refno{\ref{sec:1}}に
% 示したように,アスキー文字と漢字との間には四分空きが
% 入らなければならないが,\verb+\ref+を用いた場合には
% それが入らない.これを改善するために新しく \verb+\refno+と
% いうコマンドを定義した.「\verb+節\refno{\ref{sec:1}}に+」の
% ように使う.前者が「第\ref{sec:koko}節」で
% 後者が「第\refno{\ref{sec:koko}}節」となる.
%
\item[文献参照:] 土木学会規程通り,
参照する文献番号は右上肩に片括弧付きで表示する.
以前の版では,引用側の文章と文献番号が行末で改行分離される現象が
発生していたが,2012年秋からは改善してある.さて,文中に
どうしても書きたい場合もあるかもしれない.例えば「・・は文献2)を参照し」の
ような場合である.学会論文集では
それを容認するとは考え難いが,査読者によってはこういう
表現を(規定を無視して)要望する方がおられるようだ.\verb+\textcite+を使えば
そのような「出力\textcite{1}」になるが,規定を考えると使わない方がいい.
%
\item[複数の文献参照:] 土木学会論文集で容認されるかどうかは
確認していないが,例えば数編の論文を「$\sim$」を用いて
最初と最後の番号だけで引用する希望が多いことから,
新しく \verb+\CITE{1st,last}+を使えるようにした.
例えばこのように\CITE{1,4}のようになる.
三つ以上指定するとおかしなことになるので注意.
漢字の「〜」は使っていないので英文論文でも使える.
\end{Description}

\subsection{箇条書き環境}

スペースを確保するために箇条書き
環境周辺の上下の空きを小さくした.もし通常の設定値で
箇条書き環境を使いたい場合には,大文字で始まる`{\tt Itemize}',
 `{\tt Enumerate}', `{\tt Description}'環境を用いる.
オリジナルの定義がこの大文字で始まる環境になっている.
この節と次の節ではこの大文字版(元々の箇条書き環境)を用いてある.

\subsection{謝辞}

あまりレイアウト命令を使って欲しくないので新たに \verb+\ack+を定義し,
直前の段落最後との間に1\,emだけ空けて \verb+\paragraph+命令で
謝辞を書き出せるようにした.もし1\,em以外の
空行を空けたい場合(例として1\,cm)には,
その寸法を \verb+\ack[1cm]+で定義する.


\section{よくある質問等}
\label{sec:qanda}

\def\Qitem{\def\labelenumi{\bf Q.\theenumi:}\item}
\def\Aitem{\addtocounter{enumi}{-1}\def\labelenumi{\bf A.\theenumi:}\item}
\begin{Enumerate}
\parindent=1zw
\Qitem 2行になる題目の改行箇所を指定したいのですが,
どうしたらいいでしょう.
\Aitem \verb+\title+の中の当該箇所に \verb+\\+を入れればできます.
%
\Qitem 2行になる著者所属欄の改行箇所を指定したのですが,
よくわからないエラーが出ます.
\Aitem \verb+\thanks+の中での
強制改行は \verb+\protect\\+のように改行 \verb+\\+を保護してください.
%
\Qitem 最初のページにヘッダが表示されてしまいます.どうしたら消せますか.
\Aitem 以前は受理されるまではヘッダを付けていましたので,将来の
ことも考えてその機能を
そのまま温存してあります.ヘッダを消すには,
プリアンブルの \verb+\finalversion+を有効にしてください.
ちなみに,応用力学論文集では受理された最終原稿では \verb+\AcceptedPaper+を
有効にしてください.ページ番号が消えます.
%
\Qitem 構造工学論文集の日本語キーワードが明朝体の斜体に指定してありますが
どうしたらいいですか.
\Aitem PostScriptの機能を用いて
斜めの文字を画像として表示する
マクロ \verb+\slantbox{20}{キーワード,}+を,
構造工学論文集のスタイルファイルで組んでありますので
それを使ってください.
明朝は明朝であってそれに太字や斜体があること自体がどこか間違っていると
おっしゃる方もあって,ごもっともだと思うのですが,
そう思うのは我々\LaTeX\ ユーザだけなのかもしれません.
%
\Qitem 1ページ目の段落間に余計な空行が入ります.
どうしたらいいですか.\label{item:titlepagecontrol}
\Aitem ページ半分程度しかない1ページ目の
文章スペースにうまく段落を配置できず,段落間に設置されている
バネが伸びてしまっているのです.
書き込めるページの本文部分の高さをちょっとだけ小さくすることによって,
このバネの伸び量を減らすことができるので,\verb+\titlepagecontrol{y}+を
アンコメントして
その`y'に数字(実数)を入れてみてください.1段組部と2段組部の間に文字`x'の
高さの \verb+y+倍の空行が入りますから,
文章スペースの高さをその分だけ減らすことができます.
間延びが無くなるような,できるだけ小さい値を試行錯誤で
設定しください.2ページ目以降でも,図表がページの半分近くを占める場合に
似たような症状が出ることがあります.
そのときは,その図表の高さを微妙に増減させることによって,段落間の
バネを縮めることができます.
%
\Qitem 節(章)番号を \verb+\label+でラベル定義して \verb+\ref+で引用すると,
ピリオドも表示されてしまいます.どうしたらいいですか.
\Aitem すみませんでした.最新のクラスファイルで新たに \verb+\Ref+と
いうコマンドを定義しました.それを使ってください.
ただし,\verb+\subsection+以下では普通の \verb+\ref+を使ってください.
%
\Qitem このマニュアルでは節建てと呼んでいますが,
一番上層は章ではないのですか.
\Aitem 最近学会では「章・節・項」と呼ぶようですが,
かつては章とは呼んでなかったと思います.また\LaTeX\ のコマンドが
組版の歴史的な常識だとすると,
論文等(article)では「節・小節・項・段落・小段落」だと判断しています.ただ
節番号を \verb+\label+, \verb+\ref+で相互参照するので,どちらでも
間違いは生じないでしょう.
報告書(report)や書籍(book)では「章」が用いられ,
そこではさらにその上に「部」(part)も用いられることがあります.
%
\Qitem 「三つ」や「みっつ」あるいは「3つ」のどれが適切でしょうか.
\Aitem これは好みの問題だったり編集者のこだわりだったりするように
感じています.しかし日本語読みの表現なので前二者がいいように感じますが
どうでしょう.
逆に,「1回」や「4編」は『ひとかい』とか『よっぺん』とは
読まないでしょうから,アラビア数字でもいいのではないでしょうか.
また「断面2次モーメント」等を「二次」とするかどうかという点については,
その論文中で統一されていればどちらでも構わないと思います.
ただ「1番最初の」とは書かないと思いますが,「2番目の仮定」とはしたいですね.
%
\Qitem 事務から回ってくる書類には,コンマが半角でそのうしろにも
スペースが無いようなものが多いように感じています,
半角コンマだけではいけませんか.
\Aitem 読み難くありませんか.その回ってくる書類の方が間違いです.
インターネットのウェブページ等でも間違いが多いので,どこかで勘違いされて
いると思われます.組版規則では横組みでは
あくまでも全角の区切り文字になっています.
それは,その方が読み易いからです.同じ理由で某社の`P'付きの
フォント「MS~P\mbox{}明朝」等もワードプロセサ文書には不適切だと感じます.
原稿用紙に書くときのように
同じ幅と高さで読み易くする規則が,長年の歴史を経た現行の組版の規則だと
推測されます.
ただし,新聞等の縦組の場合には二分まで縮めることが許容されていて,
それで読み難くはなっていませんが,最近四分まで縮められ,
上の文字より下の文字に読点が近づいておかしくなっている
場合もよく見られます.
%
\Qitem ワードプロセサでの文書作成のように,
一つの段落全部をエディタの1行で入力する先輩と,1文ずつ
改行をしてしまう先輩がいます.どちらがいいのでしょうか.
\Aitem コンピュータやエディタの能力が低かったときは,
後者の方が\TeX\ の処理が楽でした.
今は,かなり長い行も楽々と処理してくれますから,\TeX\ とエディタに
とってはどちらでも構いません.

しかし,論文を書くときは,
ある主張(1文か2文程度か)ごとに改行しておいた方が,
一つの段落の中で文章の順番を入れ替えたり,段落間で文章の入れ替えるのが
楽(特にマウスを使わない人にとっては)です.
また一番最初に論文を起こすときには,
メモ書きのようにキーワードを1行に一つずつ並べておいて,
そのキーワードの順番をいろいろ考えたあと,
個々のキーワード周辺に肉付けをして文を作るという方法を
とることができます.改行がどこにあっても出来上がりには違いが無いからです.
つまり文章の論理構成のみに集中して作文できるというメリットが\TeX\ の
入力方法にはあります.
%
\Qitem \%一文字だけのコメント行と空行に,何か違いがありますか.
\Aitem 細かいことを言えば違いがありますが,それとは関係無く,
基本的に\%一文字だけのコメント行は無い(空行の)方が,
モニター上で眺めたときに文章の論理が明確になります.
例えば節題目周辺は

\renewcommand{\baselinestretch}{.74}
\begin{verbatim}
     ・・・前の節の段落最後.
     [空行]
     \subsection{新しい節}
     [空行]
     この節の書き出し・・・
\end{verbatim}
\renewcommand{\baselinestretch}{1}
とした方が読み易く(直感的な文章構成の把握が楽に)なります.
%
\Qitem 区切り文字が行末の外には出ないようですが,詰め込まなくても
いいのでしょうか.
\Aitem 句読点のような文字を約物と呼ぶようです.
今でも文庫や新聞によっては約物が行末の外に出る(ぶら下がる)ように
なっていたりします.原稿用紙で作文したときに,行末の句読点だけは
升目の外に書いてもよかったと思いますが,これがぶら下がりです.
しかしp\LaTeX\ ではぶら下がりを避けるようです.
これに対しNTT J\TeX\ では
ぶら下がりを標準にしているようですが,きちんとしたぶら下がりには
なっていないので嫌がる人もいるようです.
論文集の場合は気にする必要はありません.
%
\Qitem ローマ数字IVや丸入り数字\,\MARU{16}\,を表すのに当該の
全角フォントを使ってはいけないと聞きましたが,dviファイルでは
きちんと表示されています.使っていいですか.
\Aitem この全角フォントが某社特有の(環境依存の)フォントなので,
一般には日本語\TeX\ では採用していませんでした.
ローマ数字は半角の`I'や`V'を使えば済みますし,丸入り数字は
このクラスファイルで定義したマクロを使えば表示できます.
ただp\LaTeX\ では使えるようになっているためにdviファイルで
表示できるようですが,pdf化の方法によってはこれが空白かゲタに
なってしまいます.
使わない方がいいです.
%
\Qitem 結論の節で,頭に数字は付けましたが箇条書きにしないで項目を
列挙したところ,査読者に箇条書きにするように言われました.
どちらでもいいのではないでしょうか.
\Aitem 最近役所から来る書類でもよく見られますが,例えば

\noindent\mbox{}\hfill
\begin{minipage}[t]{.4\textwidth}
1)まずはじめに,あれこれ
どうなりました.\\
2)次にこれをどうこうしてみたら,そうなってしまって,
結局こうなってしまいました.\\
そこでこれがわかりました.
\end{minipage}
\hfill
\begin{minipage}[t]{.4\textwidth}
\begin{enumerate}
\item[1)] まずはじめに,あれこれ
どうなりました.
\item[2)] 次にこれをどうこうしてみたら,
そうなってしまって,こうなってしまいました.
\end{enumerate}
そこでこれがわかりました.
\end{minipage}
\hfill\mbox{}

\medskip
\noindent
という風に論文が書かれていた場合,
特に最後の「そこで・・・わかりました.」が全体の結論なのか
項目2の結論なのか,左のような書き方だとそれが不明確です.
しかし右はその論理が明確です.\TeX\ は,論理構成を明確にするための
組版ソフトです.箇条は箇条書きにするのが論理的には正しいのではないでしょうか.
科学論文は読み手が自由に理解するべきものとして書かれるのではなく,
誤解無く確実に意図が通じるように書かれるべきものだからです.
%
\Qitem 箇条書きの中の段落に字下げが入らないのですが,どうしたらいいですか.
\Aitem 箇条に段落を入れるのは望ましくないとは思いますが,
どうしても複数の段落にしたい場合には自分で
字下げの大きさを指定しなければなりません.

\renewcommand{\baselinestretch}{0.75}\small\normalsize
\begin{verbatim}
     \begin{enumerate}
     \parindent=1zw        % 字下げ幅は全角1文字幅分
     \item[項目A:] その説明の中に空行があれば段落になります
\end{verbatim}
\renewcommand{\baselinestretch}{1}\small\normalsize
とすれば,
二つ目の段落から全角1文字幅分の字下げが入ります.
このマニュアルでも使っています.
%
\Qitem \TeX\ の英数字CMフォントはちょっと大きめで間延びしていて
好きになれないのですが.
\Aitem `{\tt txfonts}'というfreewareの
フォントパッケージを用いると,Times Romanに
近いフォントで,少しだけですがコンパクトになります.
お試しください.
\begin{verbatim}
     \usepackage[varg]{txfonts}
\end{verbatim}
のようにオプションの`{\tt varg}'を付すと,数式の`$v$'や`$w$'が丸くなります.
ちなみに,このマニュアルで用いています.
%
\Qitem 単位は斜体ではいけないのですか.
\Aitem 組版規則ではローマンです.数式環境に含めてしまうと
数式斜体になるので注意してください.

\renewcommand{\baselinestretch}{0.75}\small\normalsize
\begin{verbatim}
     \begin{equation}
       f=12.3\times s + 4.56, \qquad \mbox{$s$をkN/mm$^2$で算定した場合}
     \end{equation}
\end{verbatim}
\renewcommand{\baselinestretch}{1}\small\normalsize
のように,\verb+\mbox+ の中に入れてください.
文中では「78.9\,m/s$^2$」(\verb+78.9\,m/s$^2$+)
「$\sigma=12.3$\,MN/mm$^2$」(\verb+$\sigma=12.3$\,MN/mm$^2$+)と
しますが,
数式環境を等号までに留めて \verb+$\sigma=$12.3\,MN/mm$^2$+と
してしまうと「$\sigma=$12.3\,MN/mm$^2$」のように文字間が詰まって
読み難くなるので,
数字も数式環境内に含めます.
また単位の前の小さいスペースも忘れないように.
ただし,「\degC」と「\%」の前にはスペースは入れません.
%
\Qitem 単位のリットルはLでいいですか.
\Aitem \TeX\ の記号には$\ell$ (\verb+\ell+)がありますが,
全世界的にはこの記号は
標準ではないという意見の方が多数のようです.
同じ分野の国際専門誌の慣習に合わせてください.
%
\Qitem AMSのスタイルファイルを併用してもいいですか.
\Aitem それは大丈夫です.レイアウトを制御するような
設定は無いようですので,使い慣れたAMSの数式環境を用いるのは構いません.
ただし,ひょっとしたら数式環境前後のスペースや,
タブ前後のスペースがさほど詰まらないかもしれません.
学会のクラスファイルでは,\LaTeX\ 標準の数式環境のスペースを少なめに
再定義してあります.
%
\Qitem 本文中の分数で$a/b+c$のように,単位と同じような
スラッシュを使った表現でもいいですか.
\Aitem 工学系でも分野によっては,
これは誤解を生むので避けるべきだとするところが
あるようです.ひょっとすると上記を$\frac{a}{b+c}$と読む可能性もあるからです.
ここは$\slfrac{a}{b}+c$のように \verb+\slfrac+を用いる等とした方が
望ましいでしょう.
%
\Qitem 本文中に分数を入れているのですが$\frac{a}{b+c}$のように
小さい文字になってわかり難いです.どうしたらいいですか.
\Aitem クラスファイルで定義した \verb+\dfrac+を
使えば$\dfrac{a}{b+c}$(\verb|\dfrac{a}{b+c}|)のようになります.
また積分記号や総和記号も大きくしたい場合には,
その都度 \verb+\displaystyle+の宣言をしてください.

例: $\displaystyle\int f \d z$(\verb+\displaystyle\int f \d z+),
$\displaystyle\sum_{n=1}^N a_n$(\verb+\displaystyle\sum_{n=1}^N a_n+)
%
\Qitem 複数行の式の2行目の頭の符号がその後ろの記号に
近づき過ぎているように見えます.
\Aitem \LaTeX\ についてのどの参考書にも書いてありますが,
それは単項演算子になっているからです.

\renewcommand{\baselinestretch}{0.75}\small\normalsize
\begin{verbatim}
     \begin{eqnarray}
       f(x) & = & \int_0^\ell g(z) \d z \nonumber \\
            && \mbox{} - h_n(x) \label{label}
     \end{eqnarray}
\end{verbatim}
\renewcommand{\baselinestretch}{1}\small\normalsize
のように \verb+\mbox{}+を挿入して二項演算子にしなければいけません.
%
\Qitem 英文を数式環境に入れると数式斜体になるのでよくないと言われていますが,
全角文字は正しく表示されます.特に問題は無いですよね.
\Aitem これは日本語\TeX で緩和したのではないかと推測しています.
しかし数式環境に入れる説明文に英数字(半角)が入る可能性もありますから,
英文論文も執筆するかもしれない者としての習慣としては,
全角文字であっても \verb+\mbox{}+ の中に入れた方がいいと思います.
%
\Qitem 複数の文献を引用するのが \verb+\cite{1}$^,$\cite{2}+のように
面倒です.簡単にはなりませんか.
\Aitem 複数の文献は \verb+\cite{1,2}+のようにして引用\cite{1,2}します.
%
\Qitem 複数の文献を「$\sim$」を使って引用したいのですが.
\Aitem それは \verb+\CITE{first,last}+のようにして
引用\CITE{1,5}できますが,3編くらいまでなら普通に \verb+\cite{1,2,3}+の
ようにした方\cite{2,3,4}がいいのではないでしょうか.
%
\Qitem 本文中で著者名を書いて文献を引用したのですが,査読者から,2名の著者の
場合は2名を並べるように指示されました.文献リストとの対応に唯一性が
あれば問題無いと思うのですが,1名だけでは駄目でしょうか.
\Aitem そういう投稿規定があるにも拘らず守らないということは,2番目の
著者に失礼だとは思いませんか.
本来なら全員を文中に並べるべきところを,3名以上の場合は冗長になるので
省略したと考えて欲しいところです.
また,ある研究機関の人物を中心にしたグループによる複数の文献を,
その中心人物の名前のみを使って「中心人物名ら\cite{1,2}による研究に
よれば」という引用も適切ではありません.
さらに参考文献リストで一部の著者名を省略することは許されていません.
%
\Qitem 文献を引用した語のところで改行されて文献番号が次の行頭に
配置されてしまいます.
\Aitem 申し訳ありません.それは古い版のスタイルファイル・クラス
ファイルを使っているからです.最新のものと置き換えてください.
多分直るはずです.
%
\Qitem 図表そのものはセンタリングするようですが,`{\tt center}'環境は
どこまでを含めればいいですか.
\Aitem \verb+\caption{title}+は自動的にセンタリングされるので,
著者がセンタリングするのは図表そのものだけでいいので

\renewcommand{\baselinestretch}{0.75}\small\normalsize
\begin{verbatim}
     \begin{figure}
     \begin{center}
          【図】
     \end{center}
     \vspace*{-4mm}
     \caption{図の説明文}\label{fig:model}
     \end{figure}
\end{verbatim}
\renewcommand{\baselinestretch}{1}\small\normalsize
のようにします.
そのとき,図表とキャプションの間に余計な空行が`{\tt center}'環境に
よって挿入さるので,
その部分には \verb+\vspace*{-4mm}+のような調整を入れると
無駄が少なくなります.
あるいは`{\tt center}'環境は使わずに

\renewcommand{\baselinestretch}{0.75}\small\normalsize
\begin{verbatim}
     \centering 【図】    % この行以下がすべてセンタリングされてしまいます
     \mbox{}\hfill 【図】 \hfill\mbox{}
\end{verbatim}
\renewcommand{\baselinestretch}{1}\small\normalsize
のどちらかにするとキャプションとの間のスペースは小さくなります.
勧めるわけではありませんが.
%
\Qitem 幅が小さい図なので,`{\tt wrapfig}'等のスタイルファイルを
使って文章を図の横に回り込ませてもいいですか.
\Aitem 読み難くなるのでやめてください.
図表は1段幅か2段幅のどちらかにするのが学会規定です.
%
\Qitem フロートを最初の引用箇所付近に確実に
配置するために \verb+[h]+オプションか,あるいは`{\tt here}'等の
スタイルファイルを使って,フロートの強制的な
位置決めをしたいのですが,構いませんか.
\Aitem それをすると本文の組版が制限されることになって,
読み難い文章になりますからやめてください.土木学会規定では
フロートは原則としてページの上方にまとめて配置することになっています.
それが困難な場合には別ページに回されます.
フロートも前から順に配置が決まりますから,原稿が完成したら,
前にあるフロート(図等)から順に,本文で初めて引用する箇所と
同じページに配置されるように,引用箇所の{\bf 少しだけ前の位置}に
配置していきます.それが難しいときには,論文を見開きにした
状態(左が偶数ページで右が奇数ページ)で最初の引用箇所とその
被引用フロートの両方が見えるように努めるのが
望ましい(最近は冊子体ではなく片面印刷で論文を読むのでどうでも
いいのかもしれませんが)です.
しかしそれも困難な場合には,直前直後のページに
配置しても問題はありません.
%
\Qitem 図表等が思ったところに配置されません.段落の途中に図表を配置しても
大丈夫ですか.
\Aitem 基本的にフロートは,
段落と段落の間に上下に空行を入れて配置します.
その方が段落途中に変な文字間隔等が生じる恐れが無くなります.
どうしても微調整して,
特定のページに来るように努力したい場合には,例えば段落中の
全角で終わる文章と全角で始まる文章の間に

\renewcommand{\baselinestretch}{.74}
\begin{verbatim}
     ・・・文章の終わりが全角で
     %
     \begin{figure}
     【図】
     \end{figure}
     %
     次の文章の始まりも全角で・・・
\end{verbatim}
\renewcommand{\baselinestretch}{1}
のようにしてもいいですが,それでもなお,フロートの大きさに
よっては,希望するページへの配置は困難な場合が普通です.
また図の挿入の仕方によっては,
この上の例の \verb+\begin{figure}+から \verb+\end{figure}+までの
すべての行末に`\%'を付した方が安全な場合もあります.
あまり無理せず,見開きで図表が眺められるような配置で十分です.
%
\Qitem 図と本文の間に余分なスペースがあるので,\verb+\end{figure}+の
次の行に \verb+\vspace*{-5mm}+と付けてみたのですが,縮まりません.
またページの半分くらいが図で覆われた場合,本文の段落間が異様に空きます.
どうしたらいいですか.
\Aitem 図表はフロートと呼ばれる通り,\verb+\begin+から \verb+\end+まで
があたかも四角の浮きのようなものとして扱われて,ページの中を浮遊します.
ですから,`{\tt figure}'環境の外(上の質問では \verb+\end+のあと)に付けた
空間制御は図の外の本文部分の制御になってしまいます.
このような場合は,あたかも四角い浮きが小さい振りをするために,\verb+\end+の
前に空間制御を置いてください.
ただし,あまり近づけると本文が読み難くなりますから,本当はこういう制御は
しない方がいいのです.

また前述の \verb+\titlepagecontrol+についての説明でも述べたように,
本文のスペースがページの半分近くになると,段落をうまく配置できずに
段落間のバネが伸びきってしまうことがあります.
そのときは,図を若干大きめにする等して本文部分をさらに小さくするか,
あるいは逆に図を小さくする等で本文のスペースを変更してみてください.
それよりも,図の配置を変更する方が効果があるかもしれません.
これには試行錯誤しかありません.
%
\Qitem 横長の大きな図表を,ページ横向きに配置することはできませんか.
\Aitem できないことはありませんが,pdfファイルを作るのが
面倒になります.工夫をして,そうならないようにした方がいいですし,
学会が受理するかどうかは分かりません.が,
一応やり方を書いておきます.
まずプリアンブルで`{\tt portland}'というスタイルファイルを
\begin{verbatim}
     \usepackage{portland}
\end{verbatim}
で読み込みます.
そして横向きの図表を最後の \verb+\lastpagesettings+の行の下に
(例は表)

\renewcommand{\baselinestretch}{0.75}\small\normalsize
\begin{verbatim}
     \onecolumn
     \landscape
     \begin{table}
     .... 横長の表を定義
     \end{table}
\end{verbatim}
\renewcommand{\baselinestretch}{1}\small\normalsize
のようにして最終ページの後に配置します.
これでこの表のページだけ横長になります.
ページ番号の位置もページの横になってしまいます.
このようにしてpdf化して最後のページだけ削除して一旦保存します.
次にlandscapeモードでpdf化して最後のページだけを一旦保存し,
それをその前までのファイルと統合すれば完成します.
ただ引用の順に図表番号を付すことができません.

本文の途中でうまくページが途切れるなら,\verb+\onecolumn+,
 \verb+\landscape+のあとに
横長のページを記述して,その後 \verb+\twocolumn+, \verb+\portrait+で
元の縦長に戻せば,本文途中に横向きのページを作ることも可能で,
図表番号にも規則違反は生じませんが,
そのような都合のいい途切れ箇所が本文中にあるとは限りませんから推奨しません.
挿入前には都合のいい途切れ箇所に見えたとしても,挿入した途端に無駄な
空行が入ることが多いからです.とりあえず,
ページ番号の配置やラベル付けをごまかす方法を`{\tt portland.tex}'に
示し,pdf化の方法を`{\tt portland.bat}'に定義し,
そのようにして作った例を`{\tt portland.pdf}'に示しておきました.
いずれにしても,学会で受理してくれるかどうかはわかりませんが,
何かしら書類を作る場合には参考になるかもしれません.
%
\Qitem 相互参照機能で図を引用していますが,変な数値だったり
違う図の番号になったりします.どこが間違っているのでしょうか.
\Aitem \verb+\label{label}+が \verb+\caption{text}+の
すぐうしろに配置されていない可能性があります.
番号は \verb+\caption+コマンドによって発行されるからです.
%
\Qitem 謝辞見出しが全角1文字幅だけ字下げされてしまいます.
どうしたらいいですか.
\Aitem 新しいクラスファイルでは,
著者が直接謝辞見出しを出さないでもいいように,\verb+\ack+という
コマンドを定義しました.単に

\renewcommand{\baselinestretch}{0.75}\small\normalsize
\begin{verbatim}
     \ack 本研究は・・・      % \ack の次はもちろん半角スペースです
\end{verbatim}
\renewcommand{\baselinestretch}{1}\small\normalsize
としてください.すぐ上の段落末との間のスペースを
変更したい場合には \verb+\ack[15mm]+のようにも指定できます.
デフォルトでは1\,emだけの空行が入ります.
%
\Qitem 文献リストの箇条のインデントが揃ってないようなのですが,
どこで間違っているのでしょう.
\Aitem もしかすると,10編以上の
文献リストに対して \verb+\begin{thebibliography}{5}+のようにしていませんか.
この`5'という数字はインデント幅のための情報です.10編未満なら
これでいいのですが,10編以上100編未満の
ときには \verb+\begin{thebibliography}{99}+の
ように2桁(`99'である必要はありません)にします.
%
\Qitem 文献リストを論文毎に作るのではなく,何かデータベースのような
ファイルを管理しておくと,文献リストを簡単に作ることができると
聞いたのですが,どういう方法なのでしょうか.
\Aitem それは\BibTeX\ (\JBibTeX\ が日本語対応)というシステムの利用です.
あるフォーマットで文献のデータベースをファイルにしておくと,
論文中からそのデータベースを参照することによって文献リストを
作成してくれます.
データベースのサンプルとやり方の例を`{\tt bibtex-j.pdf}'に示して
おきました.
一つ注意して欲しいのは,この土木学会クラスファイル類で定義した
複数の文献引用コマンド \verb+\CITE+を使った場合には,この方法では
正しい文献リストは作れないということです.当たり前ですけどね.
なお,\BibTeX\ 用のスタイルファイル`{\tt jsce.bst}'(やっつけ仕事で
すみません)が,ほぼ
土木学会形式になる文献リストを生成しますので,それ以外の論文集への
投稿の場合には,それぞれのスタイルで生成する必要があります.
例えばElsevierの配布アーカイブにもいくつかの書式が定義されていますので,
参考にしてください.
%
\Qitem 文末の英文タイトル・概要部が本文に近づき過ぎてしまうので,
少し離したいのですがどうしたらいいでしょう.
\Aitem \verb+\lastpagecontrol{5cm}+にオプションの空白設定を
加えて,\verb+\lastpagecontrol[8mm]{5cm}+とすれば,8\,mm離すことができます.
%
\Qitem 最終ページに英文タイトル・概要部が収まらず,次のページに
表示されます.それは構わないのですが,その上方に少しだけ空行を
入れたいのですがどうしたらいいでしょう.
\Aitem \verb+\lastpagesettings+の
オプションで \verb+\lastpagesettings[1cm]+とすれば1\,cmの空白が入ります.
%
\end{Enumerate}

\section{pdfファイルにするときの注意}

フォントを埋め込まなければならないため,`{\tt dvipdfmx}'を用いる場合には
\begin{verbatim}
     dvipdfmx -f msmingoth.map -f dlbase14.map filename.dvi
\end{verbatim}
とする.
もし\TeX システムが古い場合には`{\tt msmingoth.map}'が無いかもしれない.
そのときは
\begin{verbatim}
     dvipdfmx -f msembed.map -f dlbase14.map filename.dvi
\end{verbatim}
とする.
簡単のためにバッチファイル`{\tt makepdf.bat}'を作っておいたので,
オプション無しで
\begin{verbatim}
     makepdf filename
\end{verbatim}
とすればいい.しかし,このようにしても中に読み込む図や写真の
フォントが埋め込まれていない場合もあるので注意する.
図を作成する際にも注意が必要である.Acrobat Readerでpdfファイルを
読み込み,コントロールキーとDを同時に押して表示されるフォント情報で,
すべてのフォントが埋め込まれていることを確認のこと.



\begin{thebibliography}{1}
\bibitem{honda} 本多勝一:
 \newblock 日本語の作文技術,
 \newblock 朝日文庫, 1999.
\bibitem{rakuraku} 野寺隆志:
 \newblock 楽々\LaTeX\ 第2版,
 \newblock 共立出版, 1994.
\bibitem{1} Hill, R.: A self-consistent mechanics of composite materials,
 \newblock {\em J. Mech. Phys. Solids,} Vol.13, pp.213-222, 1965.
\bibitem{2} Blevins, R.D.:
 \newblock {\em Flow-Induced Vibration,} 2nd ed.
 Van Nostrand Reinhold, New York, 1990.
\bibitem{3} Karniadakis, G.E., Orszag, S.A. and Yakhot, V.:
 Renormalization group theory simulation of
 transitional and turbulent flow over a backward-facing step,
 \newblock {\em Large Eddy Simulation of Complex Engineering and
 Geophysical Flows,} Galperin, B. and Orszag, S.A. eds.
 Cambridge University Press, Cambridge, pp.159-177, 1993.
\bibitem{4} Y.~C.~Fung:
 \newblock {\it Foundations of Solid Mechanics,}
 \newblock Prentice-Hall Inc., 1965;
 \newblock 大橋義夫, 村上澄男, 神谷紀生 共訳:
 \newblock 固体の力学/ 理論, 
 \newblock 培風館, 1970.
\bibitem{5} 土田建次,木村 一:
 \newblock 版下原稿スタイルフォーマットの作成について,
 \newblock 土木学会論文集, No.333/II-99, pp.20-33, 1994.
\end{thebibliography}

\lastpagesettings

\end{document}
